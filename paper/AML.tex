\documentclass{article} % For LaTeX2e
\usepackage{nips12submit_e,times}
\nipsfinalcopy % Uncomment for camera-ready version
\usepackage{graphicx}
\usepackage{amsmath}
\usepackage{amssymb}
\usepackage{amsthm}
\usepackage{hyperref}


\newcommand{\B}[1]{{\bf #1}}
\newcommand{\Sc}[1]{{\mathcal{#1}}}
\newcommand{\R}[1]{{\rm #1}}
\newcommand{\mB}[1]{{\mathbb{#1}}}
%%%%%%%%%%%%%%%%%%%%%%%%%
% Commands copied from hart.sty
\newcommand{\bB}{\bf{B}}\newcommand{\bN}{\bf{N}}
% \newcommand{\B{1}}{\mathbb {1}}
%%%%%%%%%%%%%%%%%%%%%%%%%


% Macros added by Burke
\newcommand{\set}[2]{\left\{#1\,\left\vert\, #2\right.\right\}}
\newcommand{\one}{\bf{1}}
\newcommand{\half}{\frac{1}{2}}
\newcommand{\map}[3]{#1\,:\,#2\rightarrow #3}
\newcommand{\cS}{\mathcal{S}}
\newcommand{\cL}{\mathcal{L}}




\newtheorem{lemma}{Lemma}[section]
%\newtheorem{remark}{Remark}[section]
\newtheorem{remark}[lemma]{Remark}
\newtheorem{theorem}[lemma]{Theorem}
\newtheorem{corollary}[lemma]{Corollary}
\newtheorem{conjecture}[lemma]{Conjecture}
\newtheorem{proposition}[lemma]{Proposition}
\newtheorem{definition}[lemma]{Definition}
\newtheorem{algorithm}[lemma]{A}

\title{Sparse Coding for Dictionary Learning in Context of Image De-noising}


\author{
Dhaivat Deepak Shah\\
\texttt{ds3267@columbia.edu} \\
\And
Gaurav Ahuja\\
\texttt{ga2371@columbia.edu} \\
\And
Sarah Panda\\
\texttt{sp3206@columbia.edu} \\
}

% The \author macro works with any number of authors. There are two commands
% used to separate the names and addresses of multiple authors: \And and \AND.
%
% Using \And between authors leaves it to \LaTeX{} to determine where to break
% the lines. Using \AND forces a linebreak at that point. So, if \LaTeX{}
% puts 3 of 4 authors names on the first line, and the last on the second
% line, try using \AND instead of \And before the third author name.

\newcommand{\fix}{\marginpar{FIX}}
\newcommand{\new}{\marginpar{NEW}}

%\nipsfinalcopy % Uncomment for camera-ready version

\begin{document}


\maketitle


\begin{abstract}
Dictionary learning involves solving the following optimization problem: in 
||x α|| ||α||m D, α −D 22+λ 1 where  is the input signal,   is the dictionary and   is the sparse representation of the signal. 
x D α  
The problem of image restoration has been addressed with a multitude of approaches. All the approaches to solve the optimisation problem fall under the 3 broad categories of  Relaxation (Basis Pursuit), Greedy approach(Matching Pursuit) or Hybrid methods. Our project  primarily focuses on the Relaxation methodology.
Here, both   and   are unknown.  Mairal, Julien, et al, 2009 present an online learning D α   algorithm[1] which involves two optimization problem. First, assumes the   to be available and D   minimizes over . This is known as the sparse coding problem. Second, updates the   after α D   obtaining  . α  
Mairal, Julien, et al, 2009 use LARS[2]  to solve the sparse coding problem. We propose to  compare the performance of the online dictionary learning algorithm by solving the sparse coding  problem using methods[1][5] like feature­sign [3], FISTA[4], Interior point, Sequential Shrinkage or  Iterative Shrinkage methods and Stochastic Gradient Descent in the context of image  restoration.
\end{abstract}

\vspace{-.2cm}
\section{Introduction}
\vspace{-.2cm}
Introduction : Problem of Image denoising

\section{Intro to Dictionary learning}
Intro to Dictionary learning
  -	KSVD-  general KSVD explaination
  -	Online Dictionary Learning

\section{KSVD}
KSVD for learning dictionaries

\section{Sparse Coding}
Sparse coding problem explained in deep and ways to approximate the sparse code
  -	Basis pursuit
  -	Matching pursuit 


\section{Summary of sparse coding techniques used:}
\subsection{FISTA}
  -	FISTA
\subsection{MP}
  -	MP
\subsection{OMP}
  -	OMP
\subsection{ALM}
  -	ALM
\subsection{Feature Sign}
  -	Feature Sign
\subsection{L1LS}
  -	L1LS

\section{Experimental Setup}

\section{Findings}

\section{Analysis}

\section{Conclusion}


\nocite{*}
\bibliographystyle{plain}
\bibliography{AML}

\newpage
\section{Appendix}


\subsection{Appendix-1}
\label{sec:Appendix1}
 

\subsection{Appendix2}
\label{Appendix2}



\end{document}
